% LaTeX template for Artifact Evaluation V20201122
%
% Prepared by Grigori Fursin with contributions from Bruce Childers,
%   Michael Heroux, Michela Taufer and other colleagues.
%
% See examples of this Artifact Appendix in
%  * SC'17 paper: https://dl.acm.org/citation.cfm?id=3126948
%  * CGO'17 paper: https://www.cl.cam.ac.uk/~sa614/papers/Software-Prefetching-CGO2017.pdf
%  * ACM ReQuEST-ASPLOS'18 paper: https://dl.acm.org/citation.cfm?doid=3229762.3229763
%
% (C)opyright 2014-2022
%
% CC BY 4.0 license
%
\newcommand{\ourdoi}{\url{http://doi.org/TODO-TODO-TODO-DOI}} 

\documentclass{sigplanconf}

\usepackage{hyperref}
\usepackage{minted}

\begin{document}

\special{papersize=8.5in,11in}

%%%%%%%%%%%%%%%%%%%%%%%%%%%%%%%%%%%%%%%%%%%%%%%%%%%%
% When adding this appendix to your paper, 
% please remove above part
%%%%%%%%%%%%%%%%%%%%%%%%%%%%%%%%%%%%%%%%%%%%%%%%%%%%

\appendix
\section{Artifact Appendix}

%%%%%%%%%%%%%%%%%%%%%%%%%%%%%%%%%%%%%%%%%%%%%%%%%%%%%%%%%%%%%%%%%%%%%
\subsection{Abstract}

{\em Obligatory}

\subsection{Artifact check-list (meta-information)}

% {\em Obligatory. Use just a few informal keywords in all fields applicable to your artifacts
% and remove the rest. This information is needed to find appropriate reviewers and gradually 
% unify artifact meta information in Digital Libraries.}

{\small
\begin{itemize}
  \item {\bf Program: } The code repository for our framework along with the test suite.
  \item {\bf Compilation: } Python for running the plotting scripts and the Lean4 toolchain, downloaded via \texttt{elan}
  \item {\bf Run-time environment: } Any operating system that supports Docker.
  \item {\bf Hardware: } Any x86-64 machine.
  \item {\bf Output: } Key theorems of the paper will be built and shown to have no unsound axioms.
  \item {\bf How much disk space required (approximately)?: } 10GB
  \item {\bf How much time is needed to prepare workflow (approximately)?: } 1hr
  \item {\bf How much time is needed to complete experiments (approximately)?: } 1hr
  \item {\bf Publicly available?: } Yes
  \item {\bf Code licenses (if publicly available)?: } MIT
  \item {\bf Archived (provide DOI)?: } \ourdoi{}
\end{itemize}
}

%%%%%%%%%%%%%%%%%%%%%%%%%%%%%%%%%%%%%%%%%%%%%%%%%%%%%%%%%%%%%%%%%%%%%
\subsection{Description}

\subsubsection{Hardware dependencies}

None.

\subsubsection{Software dependencies}

Docker is necessary to run our artifact. The Docker image has all dependencies needed to compile our framework with Lean4.


%COMMENT% %%%%%%%%%%%%%%%%%%%%%%%%%%%%%%%%%%%%%%%%%%%%%%%%%%%%%%%%%%%%%%%%%%%%%
%COMMENT% \subsection{Installation}
%COMMENT% 
%COMMENT% {\em Obligatory}

%%%%%%%%%%%%%%%%%%%%%%%%%%%%%%%%%%%%%%%%%%%%%%%%%%%%%%%%%%%%%%%%%%%%%
\subsection{Experiment workflow}

Access the docker image \texttt{opencompl-ssa} from (\ourdoi{}), and then run:

\begin{minted}[fontsize=\scriptsize]{text}
$ docker load -i opencompl-ssa.docker
$ docker run -it siddudruid/opencompl-ssa
# | This clears the build cache,
# | fetches the maths library from the build cache,
# | and builds our framework.
$ cd /code/ssa && lake clean && lake exe cache get && lake build
\end{minted}


Upon running \texttt{make -j4 test}, the test output is printed to \texttt{stdout}.
The scripts \texttt{speedup-time.py} and \texttt{speedup-rgn-time.py},
produce PDFs \texttt{speedup-time.pdf} and \texttt{speedup-rgn-time.pdf}
in the directory \texttt{/code/lz/test/lambdapure/compile/bench/}:

\begin{minted}[fontsize=\footnotesize]{text}
/code/ssa/related-work/alive/compile/bench/speedup-time.pdf
\end{minted}

To open the pdf file, keep the container running, and in another
shell instance, use the \texttt{docker cp}
command to copy files from within the container out to the host:

\begin{minted}[fontsize=\footnotesize]{text}
$ docker container ls # find   ID
$ docker cp <CONTAINERID>:<PATH/INSIDE/CONTAINER> \
            <PATH/OUTSIDE/CONTAINER>
\end{minted}
For more about \texttt{docker cp}, please see:
(\url{https://docs.docker.com/engine/reference/commandline/cp/})


%%%%%%%%%%%%%%%%%%%%%%%%%%%%%%%%%%%%%%%%%%%%%%%%%%%%%%%%%%%%%%%%%%%%%
\subsection{Evaluation and expected results}

On running \texttt{lake build}, the build succeeds with no errors.
On applying the path \texttt{patch -p1 < axioms.diff} and rebuilding with \texttt{lake build},
one should see the axioms that are used by each theorem. Check that these do not use \texttt{sorry},
by running \texttt{lake build 2>\&1 | grep Axioms} and verify by looking that `sorry` is never on this list.


%%%%%%%%%%%%%%%%%%%%%%%%%%%%%%%%%%%%%%%%%%%%%%%%%%%%%%%%%%%%%%%%%%%%%
\subsection{Full Workflow Example}

\begin{minted}[fontsize=\tiny]{text}
$ cd /code/ssa
$ lake build # builds successfully.
$ cd test/bruteforce-correctness && ./run.sh # brute forcs and runs consistency check with alive semantics upto bitwidth < 4.
+ rm 'generated*'
rm: cannot remove 'generated*': No such file or directory
+ true
+ ./llvm.py
running 'opt-15 -S generated-llvm.ll -instcombine -o generated-llvm-optimized.ll'
stdout:
None
---
stderr
None
---
+ cd ../../
+ rm -f 'generated*'
+ lake build ssaLLVMEnumerator
Build completed successfully.
+ ../../.lake/build/bin/ssaLLVMEnumerator
+ diff generated-llvm-optimized-data.csv generated-ssa-llvm-semantics.csv
+ diff /dev/fd/63 /dev/fd/62
++ awk -F, '$2 == 4' generated-ssa-llvm-semantics.csv
++ awk -F, '$2 == 4' generated-ssa-llvm-syntax-and-semantics.csv
++ sort -t, -k1,1
++ sort -t, -k1,1
root@69c3097d1b88:/code/ssa/test/bruteforce-correctness#
\end{minted}

See that the diff is empty, and that we generate the exact same results.
One can view the results by \texttt{cat} ing the generated semantics CSV file.

%%%%%%%%%%%%%%%%%%%%%%%%%%%%%%%%%%%%%%%%%%%%%%%%%%%%
% When adding this appendix to your paper, 
% please remove below part
%%%%%%%%%%%%%%%%%%%%%%%%%%%%%%%%%%%%%%%%%%%%%%%%%%%%

\end{document}

