% LaTeX template for Artifact Evaluation V20201122
%
% Prepared by Grigori Fursin with contributions from Bruce Childers,
%   Michael Heroux, Michela Taufer and other colleagues.
%
% See examples of this Artifact Appendix in
%  * SC'17 paper: https://dl.acm.org/citation.cfm?id=3126948
%  * CGO'17 paper: https://www.cl.cam.ac.uk/~sa614/papers/Software-Prefetching-CGO2017.pdf
%  * ACM ReQuEST-ASPLOS'18 paper: https://dl.acm.org/citation.cfm?doid=3229762.3229763
%
% (C)opyright 2014-2022
%
% CC BY 4.0 license
%

\documentclass{sigplanconf}

\usepackage{hyperref}
\usepackage{minted}

\usepackage[verbose]{newunicodechar}
\newunicodechar{Γ}{\ensuremath{\Gamma}}
\newunicodechar{⊢}{\ensuremath{\vdash}}
\newunicodechar{▸}{\ensuremath{\blacktriangleright}}
\newunicodechar{∅}{\ensuremath{\emptyset}}
\newunicodechar{α}{\ensuremath{\alpha}}
\newunicodechar{β}{\ensuremath{\beta}}
\newunicodechar{δ}{\ensuremath{\delta}}
\newunicodechar{Δ}{\ensuremath{\Delta}}
\newunicodechar{ϵ}{\ensuremath{\epsilon}}
\newunicodechar{τ}{\ensuremath{\tau}}
\newunicodechar{ε}{\ensuremath{\epsilon}}
\newunicodechar{σ}{\ensuremath{\sigma}}
\newunicodechar{Σ}{\ensuremath{\Sigma}}
% \newunicodechar{α}{\ensuremath{\alpha}}
\newunicodechar{∈}{\ensuremath{\in}}
\newunicodechar{∧}{\ensuremath{\land}}

\newunicodechar{₀}{\textsubscript{0}}
\newunicodechar{₁}{\textsubscript{1}}
\newunicodechar{₂}{\textsubscript{2}}
\newunicodechar{₃}{\textsubscript{3}}
\newunicodechar{ₙ}{\textsubscript{n}}
\newunicodechar{ₘ}{\textsubscript{m}}
\newunicodechar{ₕ}{\textsubscript{h}}
\newunicodechar{⊕}{\ensuremath{\oplus}}
\newunicodechar{∀}{\ensuremath{\forall}}
\newunicodechar{∃}{\ensuremath{\exists}}
\newunicodechar{∘}{\ensuremath{\circ}}
\newunicodechar{⟦}{\ensuremath{\llbracket}}
\newunicodechar{⟧}{\ensuremath{\rrbracket}}
\newunicodechar{ℕ}{\ensuremath{\mathbb{N}}}
\newunicodechar{ℤ}{\ensuremath{\mathbb{Z}}}

% \newunicodechar{⟦}{\ensuremath{\llbracket}}
% https://tex.stackexchange.com/questions/100966/defining-scalable-white-curly-brackets-and-and
% TODO FIXME: we gotta fix these parens!
\newunicodechar{⦃}{\ensuremath{\{\{}}
\newunicodechar{⦄}{\ensuremath{\}\}}}
\newunicodechar{⧸}{\ensuremath{/}}
\newunicodechar{⊑}{\ensuremath{\sqsubseteq}}

\begin{document}

\special{papersize=8.5in,11in}

%%%%%%%%%%%%%%%%%%%%%%%%%%%%%%%%%%%%%%%%%%%%%%%%%%%%
% When adding this appendix to your paper, 
% please remove above part
%%%%%%%%%%%%%%%%%%%%%%%%%%%%%%%%%%%%%%%%%%%%%%%%%%%%

\appendix
\section{Artifact Appendix}

%%%%%%%%%%%%%%%%%%%%%%%%%%%%%%%%%%%%%%%%%%%%%%%%%%%%%%%%%%%%%%%%%%%%%
% \subsection{Abstract}

% {\em Obligatory}

\subsection{Artifact check-list (meta-information)}

% {\em Obligatory. Use just a few informal keywords in all fields applicable to your artifacts
% and remove the rest. This information is needed to find appropriate reviewers and gradually 
% unify artifact meta information in Digital Libraries.}

{\small
\begin{itemize}
  \item {\bf Program: } The code repository for our framework along with the test suite. Note that this is already setup in the docker image.
  \item {\bf Compilation: } The Lean4 toolchain, downloaded via \texttt{elan}. Note that this is already setup in the docker image.
  \item {\bf Run-time environment: } Any operating system that supports Docker.
  \item {\bf Hardware: } Any x86-64 machine.
  \item {\bf Output: } Key theorems of the paper will be built and shown to have no unsound axioms.
  \item {\bf How much disk space required (approximately)?: } 10GB
  \item {\bf How much time is needed to prepare workflow (approximately)?: } 1hr
  \item {\bf How much time is needed to complete experiments (approximately)?: } 1hr
  \item {\bf Publicly available?: } Yes
  \item {\bf Code licenses (if publicly available)?: } MIT
  \item {\bf Archived (provide DOI)?: } 10.5281/zenodo.11519739
\end{itemize}
}

%%%%%%%%%%%%%%%%%%%%%%%%%%%%%%%%%%%%%%%%%%%%%%%%%%%%%%%%%%%%%%%%%%%%%
\subsection{Description}

\subsubsection{Software dependencies}

Docker is necessary to run our artifact.
The Docker image has all dependencies needed to compile our framework with Lean4.


%COMMENT% %%%%%%%%%%%%%%%%%%%%%%%%%%%%%%%%%%%%%%%%%%%%%%%%%%%%%%%%%%%%%%%%%%%%%
%COMMENT% \subsection{Installation}
%COMMENT% 
%COMMENT% {\em Obligatory}

%%%%%%%%%%%%%%%%%%%%%%%%%%%%%%%%%%%%%%%%%%%%%%%%%%%%%%%%%%%%%%%%%%%%%
\subsection{Experiment workflow}

Access the docker image \texttt{lean-mlir} from
\textcolor{red}{Zenodo link}.

\begin{minted}[fontsize=\scriptsize]{text}
  from folder lean-mlir: 
$ docker build -f artifact-evaluation/Dockerfile -t lean-image .
$ docker create --name temp-container lean-image
$ docker cp temp-container:/code/lean-mlir/bv-evaluation/raw-data ./docker-results
$ docker rm temp-container
\end{minted}

% \subsection{Full Workflow Example}


%%%%%%%%%%%%%%%%%%%%%%%%%%%%%%%%%%%%%%%%%%%%%%%%%%%%%%%%%%%%%%%%%%%%%
\subsection{Evaluation and expected results}

On running \texttt{lake build}, the build succeeds with no errors.

Next, run:

\begin{minted}[fontsize=\scriptsize]{text}
$ rg -g "**/*.lean" "guard_msgs in #print axioms" -C2 | grep "sorry"
\end{minted}

Grep returns no matches, which checks that all guarded theorems do not use the sorry axiom.
To manually inspect this, run:

\begin{minted}[fontsize=\scriptsize]{text}
$ rg -g "**/*.lean" "#guard_msgs in #print axioms" -C2
\end{minted}

In the output, observe the following lines, which tells us that, for example, the 
theorem \texttt{denote\_rewritePeepholeAt} depends only on the axioms 
\texttt{propext}, \texttt{Classical.choice}, and \texttt{Quot.sound}. That is, this theorem 
\emph{does not depend} on axioms such as \texttt{sorryAx} that compromise the correctness of Lean's verification.

Now, since the Lean build was successful in the previous step, and since we use standard Lean axioms, 
we are justified in our claims that we have mechanized the key theorem statements from the paper.

Below, we list the key theorems that we claim to have mechanized, and their 
guarded verification below. Please inspect the output of 
\texttt{(\$ rg -g "**/*.lean" "\#guard\_msgs in \#print axioms" -C2 | grep "sorry")}
and confirm that these guarded statements occur, with the guard containing only the axioms \texttt{propext},
\texttt{Classical.choice}, and \texttt{Quot.sound}.

\subsubsection{Core Framework Theorems}

The core correctness claim of the peephole rewriter is guarded below:

\begin{minted}[fontsize=\tiny]{text}
SSA/Core/Framework.lean
2422-
2423-/-- info: 'denote_rewritePeepholeAt' depends on axioms: [propext, Classical.choice, Quot.sound] -/
2424:#guard_msgs in #print axioms denote_rewritePeepholeAt
2425-
2426-/- repeatedly apply peephole on program. -/
--
2458-
2459-/-- info: 'denote_rewritePeephole' depends on axioms: [propext, Classical.choice, Quot.sound] -/
2460:#guard_msgs in #print axioms denote_rewritePeephole
2461-
2462-end SimpPeepholeApplier
\end{minted}

\subsubsection{Five Hardest Alive Examples}

The correctness claim of the hardest Alive rewrites that time out on an SMT solver, which we verify 
for arbitrary bitwidth:

\begin{minted}[fontsize=\tiny]{text}
SSA/Projects/InstCombine/AliveHandwrittenLargeExamples.lean
53-/--info: 'AliveHandwritten.DivRemOfSelect.alive_DivRemOfSelect' depends on
54-axioms: [propext, Classical.choice, Quot.sound] -/
55:#guard_msgs in #print axioms alive_DivRemOfSelect
56-
57-end DivRemOfSelect
--
215-/--info: 'AliveHandwritten.MulDivRem.alive_simplifyMulDivRem805' depends on axioms:
216-[propext, Classical.choice, Quot.sound] -/
217:#guard_msgs in #print axioms alive_simplifyMulDivRem805
218-
219-open Std (BitVec) in
--
302-/--info: 'AliveHandwritten.MulDivRem.alive_simplifyMulDivRem805'' depends on axioms:
303-[propext, Classical.choice, Quot.sound] -/
304:#guard_msgs in #print axioms alive_simplifyMulDivRem805'
305-
306-/-
--
357-/-- info: 'AliveHandwritten.MulDivRem.alive_simplifyMulDivRem290' depends on
358-axioms: [propext, Classical.choice, Quot.sound]-/
359:#guard_msgs in #print axioms alive_simplifyMulDivRem290
360-
361-end MulDivRem
--
416-/-- info: 'AliveHandwritten.AndOrXor.alive_simplifyAndOrXor2515' depends on
417-axioms: [propext, Classical.choice, Quot.sound] -/
418:#guard_msgs in #print axioms alive_simplifyAndOrXor2515
419-
420-/-
--
532-/-- info: 'AliveHandwritten.Select.alive_simplifySelect764' depends on axioms:
533-[propext, Classical.choice, Quot.sound] -/
534:#guard_msgs in #print axioms alive_simplifySelect764
535-
536-end Select
\end{minted}

\subsection{Paper Code Examples}

\begin{itemize}
\item \emph{Figure 1, 2} is at \texttt{SSA/Projects/FullyHomomorphicEncryption/\\PaperExamples.lean}.
\item \emph{Figure 3, 4} is at \texttt{SSA/Core/Framework.lean}.
\item The statement and proof of \texttt{denote\_rewritePeephole} can be found at 
\texttt{SSA/Core/Framework.lean}. 
\item The definitions and proofs of DCE can be found in \\
\texttt{SSA/Projects/DCE/DCE.lean}, and of CSE in \\\texttt{SSA/Projects/CSE/CSE.lean}.
\item The examples for bitvector rewrites in found at \\\texttt{SSA/Projects/InstCombine/PaperExamples.lean}.
\item The hand-written examples are found at \\\texttt{SSA/Projects/InstCombine/AliveHandwrittenLargeExamples.lean}.
\item All FHE examples can be found in \\\texttt{SSA/Projects/FullyHomomorphicEncryption/PaperExamples.lean}.
\end{itemize}

\subsection{Miscellanous Docker Usage}
To copy files for inspection from the docker container into the host,
 keep the container running, and in another
shell instance, use the \texttt{docker cp}
command to copy files from within the container out to the host:\footnote{For more about \texttt{docker cp}, please see: (\url{https://docs.docker.com/engine/reference/commandline/cp/})}

\begin{minted}[fontsize=\footnotesize]{text}
$ docker container ls # find   ID
$ docker cp <CONTAINERID>:<PATH/INSIDE/CONTAINER> \
            <PATH/OUTSIDE/CONTAINER>
\end{minted}


%%%%%%%%%%%%%%%%%%%%%%%%%%%%%%%%%%%%%%%%%%%%%%%%%%%%%%%%%%%%%%%%%%%%%



%%%%%%%%%%%%%%%%%%%%%%%%%%%%%%%%%%%%%%%%%%%%%%%%%%%%
% When adding this appendix to your paper, 
% please remove below part
%%%%%%%%%%%%%%%%%%%%%%%%%%%%%%%%%%%%%%%%%%%%%%%%%%%%

\end{document}
