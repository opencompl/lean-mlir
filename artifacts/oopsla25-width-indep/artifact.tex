  %
% Prepared by Grigori Fursin with contributions from Bruce Childers,
%   Michael Heroux, Michela Taufer and other colleagues.
%
% See examples of this Artifact Appendix in
%  * SC'17 paper: https://dl.acm.org/citation.cfm?id=3126948
%  * CGO'17 paper: https://www.cl.cam.ac.uk/~sa614/papers/Software-Prefetching-CGO2017.pdf
%  * ACM ReQuEST-ASPLOS'18 paper: https://dl.acm.org/citation.cfm?doid=3229762.3229763
%
% (C)opyright 2014-2022
%
% CC BY 4.0 license
%

\documentclass[acmlarge, nonacm]{acmart}

% \documentclass[onecolumn]{sigplanconf}

\usepackage{hyperref}
\usepackage{minted}

\newminted[script]{bash}{style=bw, bgcolor=blue!5!white, breaklines, fontsize=\footnotesize}

\usepackage[verbose]{newunicodechar}
\newunicodechar{Γ}{\ensuremath{\Gamma}}
\newunicodechar{⊢}{\ensuremath{\vdash}}
\newunicodechar{▸}{\ensuremath{\blacktriangleright}}
\newunicodechar{∅}{\ensuremath{\emptyset}}
\newunicodechar{α}{\ensuremath{\alpha}}
\newunicodechar{β}{\ensuremath{\beta}}
\newunicodechar{δ}{\ensuremath{\delta}}
\newunicodechar{Δ}{\ensuremath{\Delta}}
\newunicodechar{ϵ}{\ensuremath{\epsilon}}
\newunicodechar{τ}{\ensuremath{\tau}}
\newunicodechar{ε}{\ensuremath{\epsilon}}
\newunicodechar{σ}{\ensuremath{\sigma}}
\newunicodechar{Σ}{\ensuremath{\Sigma}}
% \newunicodechar{α}{\ensuremath{\alpha}}
\newunicodechar{∈}{\ensuremath{\in}}
\newunicodechar{∧}{\ensuremath{\land}}

\newunicodechar{₀}{\textsubscript{0}}
\newunicodechar{₁}{\textsubscript{1}}
\newunicodechar{₂}{\textsubscript{2}}
\newunicodechar{₃}{\textsubscript{3}}
\newunicodechar{ₙ}{\textsubscript{n}}
\newunicodechar{ₘ}{\textsubscript{m}}
\newunicodechar{ₕ}{\textsubscript{h}}
\newunicodechar{⊕}{\ensuremath{\oplus}}
\newunicodechar{∀}{\ensuremath{\forall}}
\newunicodechar{∃}{\ensuremath{\exists}}
\newunicodechar{∘}{\ensuremath{\circ}}
\newunicodechar{⟦}{\ensuremath{\llbracket}}
\newunicodechar{⟧}{\ensuremath{\rrbracket}}
\newunicodechar{ℕ}{\ensuremath{\mathbb{N}}}
\newunicodechar{ℤ}{\ensuremath{\mathbb{Z}}}

% \newunicodechar{⟦}{\ensuremath{\llbracket}}
% https://tex.stackexchange.com/questions/100966/defining-scalable-white-curly-brackets-and-and
% TODO FIXME: we gotta fix these parens!
\newunicodechar{⦃}{\ensuremath{\{\{}}
\newunicodechar{⦄}{\ensuremath{\}\}}}
\newunicodechar{⧸}{\ensuremath{/}}
\newunicodechar{⊑}{\ensuremath{\sqsubseteq}}

\begin{document}

\special{papersize=8.5in,11in}

%%%%%%%%%%%%%%%%%%%%%%%%%%%%%%%%%%%%%%%%%%%%%%%%%%%%
% When adding this appendix to your paper,
% please remove above part
%%%%%%%%%%%%%%%%%%%%%%%%%%%%%%%%%%%%%%%%%%%%%%%%%%%%

\appendix
\title[Artifact]{Artifact for Certified Decision Procedures for Width-Independent
Bitvector Predicates in Interactive Theorem Provers}

\maketitle

\section{Introduction}

This artifact contains the decision procedures introduced by the coresponding
paper, it aims to demonstrate that they are correct and proved correct using the
Lean proof assistant, and to demonstrate they performance.

\subsection{Correctness}

To demonstrate the correctness, we use the decision procedures to prove lemmas
and check that Lean accepts the proof, and that they do not rely on unexpected
axioms such as \texttt{sorryAx}.

\subsection{Performance}

To demonstrate their performance, we test the decision procedures on a set of
tests, one coming from LLVM and one coming from the obfuscation litterature. The
artifact contains the inputs of these benchmarks as well as scripts to run then
and generate the plots in the paper.

\section{Hardware Dependencies}

Podman or Docker are necessary to run our artifact (we tested it with the former).
%
The container image has all dependencies needed to compile our framework with Lean4.

The artifact requires a somewhat beefy machine to run: the largest evaluation
contains 10000 problems, and 2500 of the problems exercise our k-induction
solver which uses a SAT solver, which can require a large amount of RAM.

We ran our evaluation on a machine with 32 cores and 125 GB of RAM, though 32 GB
should suffice.

\section{Getting Started}

Access the docker image from \url{https://doi.org/10.5281/zenodo.15754222}.

\begin{script}
$ docker load -i oopsla25-width-indep.tar.zst
$ docker run -it oopsla25-width-indep
# Run a smaller version of the experiments, and check that the output is as expected.
$ /code/lean-mlir/artifacts/oopsla25-width-indep/test_experiments.sh
$ cat /code/lean-mlir/bv-evaluation/automata-automata-circuit-cactus-plot-data.tex
$ cat /code/lean-mlirSSA/Experimental/Bits/Fast/Dataset2/dataset2-cactus-plot-data.tex
# (Optional) View the cactus plots from within the container.
$ timg -p iterm2 /code/lean-mlir/bv-evaluation/automata-automata-circuit-cactus-plot.jpeg
$ timg -p iterm2 /code/lean-mlir/SSA/Experimental/Bits/Fast/Dataset2/automata-automata-circuit-cactus-plot.jpeg
\end{script}


\section{Step by Step Instructions}

As for the previous section, load the image.

\begin{script}
$ docker load -i oopsla25-width-indep.tar
$ docker run -it oopsla25-width-indep
\end{script}

\subsection{Checking That Our Decision Procedures Are Verified}

Check that the theorems that are automatically proved by our decision
procedures do not rely on incomplete proofs, which is materialized as the
sorryAx axiom in Lean.

This relies on the \texttt{guard\_msgs} command of Lean, which enforces that the
comment contains exactly the output of a Lean command, which is used here with
the \texttt{print axioms} command which lists all the axioms used (transitively)
by a theorem.

\begin{script}
# | This allows to check that the key theorems of our decision procedures are
# | guarded, and that they do not contain `sorry`s.
$ rg -g "SSA/Experimental/Bits/**/*.lean" "#guard_msgs in #print axioms" -C7 | grep "sorry"
# | This allows to manually inspect that the key theorems of our paper are sorry-free.
$ rg -g "SSA/Experimental/Bits/**/*.lean" "#guard_msgs in #print axioms" -C7
\end{script}


\subsubsection{Automata Decision Procedure}

The next shows that the key lemma of the automata based decision procedure relies on no axioms
except for missing hashMap lemmas in Lean:

\begin{script}
AutoStructs/FormulaToAuto.lean
1944-/--
1945-info: 'Formula.denote_of_isUniversal' depends on axioms: [hashMap_missing,
1946- propext,
1947- Classical.choice,
1948- Lean.ofReduceBool,
1949- Quot.sound]
1950--/
1951:#guard_msgs in #print axioms Formula.denote_of_isUniversal
\end{script}

Furthermore, the tactic, when used against an example, also produces no extra axioms
beyond the standard lean axioms, plus \texttt{hashMap\_missing}:

\begin{script}
Fast/Tests.lean
58-theorem check_axioms_presburger (w : Nat) (a b : BitVec w) : a + b = b + a := by
59-  bv_automata_gen (config := {backend := .automata} )
60-
61-/--
62-info: 'check_axioms_presburger' depends on axioms: [hashMap_missing,
63- propext,
64- Classical.choice,
65- Lean.ofReduceBool,
66- Lean.trustCompiler,
67- Quot.sound]
68--/
69:#guard_msgs in #print axioms check_axioms_presburger
\end{script}

\subsubsection{K-induction decision procedure}

The example below show that the key lemma,
as well as the use of the tactic on a an example theorem
do not use any axioms beyond the standard Lean inbuilt
axioms of propositional extensionality, classical choice principles, and soundness of quotient types:

\begin{script}
Fast/ReflectVerif.lean
2561-    simp
2562-
2563-/--
2564-info: 'ReflectVerif.BvDecide.KInductionCircuits.Predicate. denote_of_verifyCircuit_mkSafetyCircuit_of_verifyCircuit_mkIndHypCycleBreaking' depends on axioms: [propext,
2565- Classical.choice,
2566- Quot.sound]
2567--/
2568:#guard_msgs in #print axioms Predicate.denote_of_verifyCircuit_mkSafetyCircuit_of_verifyCircuit_mkIndHypCycleBreaking
...
Fast/Tests.lean
49-set_option trace.Bits.FastVerif true in
50-theorem check_axioms_cadical (w : Nat) (a b : BitVec w) : a + b = b + a := by
51-  bv_automata_gen (config := {backend := .circuit_cadical_verified} )
52-
53-/--
54-info: 'check_axioms_cadical' depends on axioms: [propext, Classical.choice, Lean.ofReduceBool, Quot.sound]
55--/
56:#guard_msgs in #print axioms check_axioms_cadical
\end{script}


\subsubsection{MBA Decision Procedure}

The example below show that the key lemma,
as well as the use of the tactic on a an example theorem
do not use any axioms beyond the standard Lean inbuilt
axioms of propositional extensionality, classical choice principles, and soundness of quotient types:


\begin{script}
Fast/MBA.lean
636-info: 'MBA.Eqn.forall_width_reflect_zero_of_width_one_denote_zero' depends on axioms: [propext, Classical.choice, Quot.sound]
637--/
638:#guard_msgs in #print axioms Eqn.forall_width_reflect_zero_of_width_one_denote_zero
639-
640-@[simp]
--
985-info: 'MBA.Examples.check_axioms_mba' depends on axioms: [propext, Classical.choice, Quot.sound]
986--/
987:#guard_msgs in #print axioms check_axioms_mba
988-
989-end Examples
\end{script}



\section{Full Scale Run Of Our Experiments}

Run the two large experiments and reproduce the data and graphs from the paper:

\begin{script}
# Run experiments, and check that the output is as expected.
# (Parallelism can be tweaked by changing the -j options by editing the script.)
$ /code/lean-mlir/artifacts/oopsla25-width-indep/run_experiments.sh
$ cat /code/lean-mlir/bv-evaluation/automata-automata-circuit-cactus-plot-data.tex
$ cat /code/lean-mlirSSA/Experimental/Bits/Fast/Dataset2/dataset2-cactus-plot-data.tex
# (Optional) View the cactus plots from within the container.
$ timg -p iterm2 /code/lean-mlir/bv-evaluation/automata-automata-circuit-cactus-plot.jpeg
$ timg -p iterm2 /code/lean-mlir/SSA/Experimental/Bits/Fast/Dataset2/automata-automata-circuit-cactus-plot.jpeg
\end{script}

The tex files contain data correponding to Figure 7 of the paper, and the two
graphs correspond to figures (a) and (b) page 18 of the paper as follows.

\subsubsection{Verifying the Results for LLVM rewrites (\texttt{bv-evaluation}).}
Please run:
\begin{script}
$ cat /code/lean-mlir/bv-evaluation/automata-automata-circuit-cactus-plot-data.tex
\end{script}

{\sloppypar
This should produce the following output, where the timings might change
depending on the machine that is run on. The absolute number of problems solved
should be equal, while the total time and geomean time are machine-dependent.
}

TODO: we should explain the mapping between the names of the macros and the
paper names.

\begin{script}
# $ cat /code/lean-mlir/bv-evaluation/automata-automata-circuit-cactus-plot-data.tex
\newcommand{\InstCombineNormCircuitVerifiedNumSolved}{2486} # exact match
\newcommand{\InstCombineNormPresburgerNumSolved}{2126} # exact match
\newcommand{\InstCombineBvDecideNumSolved}{7781} # exact match
\newcommand{\InstCombineNormCircuitVerifiedTotalTime}{...}
\newcommand{\InstCombineNormPresburgerTotalTime}{...}
\newcommand{\InstCombineBvDecideTotalTime}{...}
\newcommand{\InstCombineNormCircuitVerifiedGeoMean}{42ms} # machine-dependent
\newcommand{\InstCombineNormPresburgerGeoMean}{25ms} # machine-dependent
\newcommand{\InstCombineBvDecideGeoMean}{66ms} # machine-dependent
\newcommand{\InstCombineTotalProblems}{7855} # exact match
\end{script}

To view the plot data, either use \texttt{docker cp} to copy the file out of the container,
or use the installed \texttt{timg} tool to view the plot data within the container.
Use the \texttt{timg -p} option to choose the protocol to view images.
This will work on terminal emulators that support the sixel protocol.
\begin{script}
$ timg -p iterm2 /code/lean-mlir/bv-evaluation/automata-automata-circuit-cactus-plot.jpeg
\end{script}

\subsubsection{Verifying the Results for MBA (\texttt{SSA/Experimental/Bits/Fast/Dataset2}).}

Please run:
\begin{script}
$ cat /code/lean-mlir/SSA/Experimental/Bits/Fast/Dataset2/dataset2-cactus-plot-data.tex
\end{script}

{\sloppypar
This should produce the following output, where the timings might change
depending on the machine that is run on. The absolute number of problems solved
should be equal, while the total time and geomean time are machine-dependent.
}

\begin{script}
# $ cat /code/lean-mlir/SSA/Experimental/Bits/Fast/Dataset2/dataset2-cactus-plot-data.tex
\newcommand{\MBAKInductionVerifiedNumSolved}{1546} # > 1500
\newcommand{\MBAMBANumSolved}{2500} # exact match
\newcommand{\MBABvDecideNumSolved}{2500} # exact match
\newcommand{\MBAPresburgerNumSolved}{2500} #exact match
\newcommand{\MBAKInductionVerifiedTotalTime}{2h2m}
\newcommand{\MBAMBATotalTime}{6m17s}
\newcommand{\MBABvDecideTotalTime}{6h59m}
\newcommand{\MBAPresburgerTotalTime}{15h13m}
\newcommand{\MBAKInductionVerifiedGeoMean}{2.7s}
\newcommand{\MBAMBAGeoMean}{54.2ms}
\newcommand{\MBABvDecideGeoMean}{9.89s}
\newcommand{\MBAPresburgerGeoMean}{4.26s}
\end{script}

The plots are located at:
\begin{script}
$ timg -p iterm2 /code/lean-mlir/SSA/Experimental/Bits/Fast/Dataset2/automata-automata-circuit-cactus-plot.jpeg
\end{script}


\end{document}

\subsection{Artifact check-list (meta-information)}

% {\em Obligatory. Use just a few informal keywords in all fields applicable to your artifacts
% and remove the rest. This information is needed to find appropriate reviewers and gradually
% unify artifact meta information in Digital Libraries.}

{\small
\begin{itemize}
  \item {\bf Program: } The code repository for our framework along with the test suite. Note that this is already setup in the docker image.
  \item {\bf Compilation: } The Lean4 toolchain, downloaded via \texttt{elan}. Note that this is already setup in the docker image.
  \item {\bf Run-time environment: } Any operating system that supports Docker.
  \item {\bf Hardware: } Any x86-64 machine.
  \item {\bf Output: } Key theorems of the paper will be built and shown to have no unsound axioms.
  \item {\bf How much disk space required (approximately)?: } 30GB
  \item {\bf How much time is needed to prepare workflow (approximately)?: } 1hr
  \item {\bf How much time is needed to complete experiments (approximately)?: } 5hr
  \item {\bf Publicly available?: } Yes
  \item {\bf Code licenses (if publicly available)?: } MIT
  \item {\bf Archived (provide DOI)?: } 10.5281/zenodo.15754222
\end{itemize}
}

%%%%%%%%%%%%%%%%%%%%%%%%%%%%%%%%%%%%%%%%%%%%%%%%%%%%%%%%%%%%%%%%%%%%%
\subsection{Description}


%COMMENT% %%%%%%%%%%%%%%%%%%%%%%%%%%%%%%%%%%%%%%%%%%%%%%%%%%%%%%%%%%%%%%%%%%%%%
%COMMENT% \subsection{Installation}
%COMMENT%
%COMMENT% {\em Obligatory}

%%%%%%%%%%%%%%%%%%%%%%%%%%%%%%%%%%%%%%%%%%%%%%%%%%%%%%%%%%%%%%%%%%%%%
\subsection{Experiment workflow}

Access the tar file of the docker image from \url{https://doi.org/10.5281/zenodo.15754222}.

\begin{script}
$ docker load -i oopsla25-width-indep.tar
$ docker run -it oopsla25-width-indep
# | This allows to check that the key theorems of our framework are
# |guarded, and that they do not contain `sorry`s.
$ rg -g "SSA/Experimental/Bits/Fast/**/*.lean" "#guard_msgs in #print axioms" -C2 | grep "sorry"
$ rg -g "SSA/Experimental/Bits/Fast/**/*.lean" "#guard_msgs in #print axioms" -C2
# Run experiments, and check that the output is as expected.
$ /code/lean-mlir/artifacts/oopsla25-width-indep/run_experiments.sh
$ cat /code/lean-mlir/bv-evaluation/automata-automata-circuit-cactus-plot-data.tex
$ cat /code/lean-mlirSSA/Experimental/Bits/Fast/Dataset2/dataset2-cactus-plot-data.tex
# (Optional) View the cactus plots from within the container.
$ timg -p iterm2 /code/lean-mlir/bv-evaluation/automata-automata-circuit-cactus-plot.jpeg
$ timg -p iterm2 /code/lean-mlir/SSA/Experimental/Bits/Fast/Dataset2/automata-automata-circuit-cactus-plot.jpeg
\end{script}


{\sloppypar
This should produce the following output, where the timings might change
depending on the machine that is run on. The absolute number of problems solved
should be 2500 on the fast solver (MBA), and should be appreciably
large on the other solvers. The geomean time are machine-dependent,
and should match upto order of magnitude comparisons.
}

\begin{script}
# $ cat /code/lean-mlir/SSA/Experimental/Bits/Fast/Dataset2/dataset2-cactus-plot-data.tex
\newcommand{\MBAPresburgerNumSolved}{2500} # exact match
\newcommand{\MBABvDecideNumSolved}{2500} # exact match
\newcommand{\MBAKInductionVerifiedNumSolved}{1537} # > 1500
\newcommand{\MBAMBANumSolved}{2500} # exact match
\newcommand{\MBAPresburgerTotalTime}{..} # machine dependent
\newcommand{\MBABvDecideTotalTime}{..} # machine dependent
\newcommand{\MBAKInductionVerifiedTotalTime}{..} # machine dependent
\newcommand{\MBAMBATotalTime}{..} # machine dependent
\newcommand{\MBAPresburgerGeoMean}{4.6s} # machine dependent
\newcommand{\MBABvDecideGeoMean}{10.9s} # machine dependent
\newcommand{\MBAKInductionVerifiedGeoMean}{3.68s} # machine dependent
\newcommand{\MBAMBAGeoMean}{78.9ms} # machine dependent
\end{script}

To view the plot data, either use \texttt{docker cp} to copy the file out of the container,
or use the installed \texttt{timg} tool to view the plot data within the container.
Use the \texttt{timg -p} option to choose the protocol to view images.
This will work on terminal emulators that support the sixel protocol.
\begin{script}
$ timg -p iterm2 /code/lean-mlir/SSA/Experimental/Bits/Fast/Dataset2/automata-automata-circuit-cactus-plot.jpeg
\end{script}

\subsection{Miscellanous Docker Usage}
To copy files for inspection from the docker container into the host,
 keep the container running, and in another
shell instance, use the \texttt{docker cp}
command to copy files from within the container out to the host:\footnote{For more about \texttt{docker cp}, please see: (\url{https://docs.docker.com/engine/reference/commandline/cp/})}

\begin{script}
$ docker container ls # find   ID
$ docker cp <CONTAINERID>:<PATH/INSIDE/CONTAINER> \
            <PATH/OUTSIDE/CONTAINER>
\end{script}


%%%%%%%%%%%%%%%%%%%%%%%%%%%%%%%%%%%%%%%%%%%%%%%%%%%%%%%%%%%%%%%%%%%%%



%%%%%%%%%%%%%%%%%%%%%%%%%%%%%%%%%%%%%%%%%%%%%%%%%%%%
% When adding this appendix to your paper,
% please remove below part
%%%%%%%%%%%%%%%%%%%%%%%%%%%%%%%%%%%%%%%%%%%%%%%%%%%%

\end{document}

